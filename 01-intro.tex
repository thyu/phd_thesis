\chapter{Introduction}
\label{chap/intro} 



This thesis addresses the problems of object recognition and pose estimation for 3-D shapes and human actions in videos. 

\section{Motivation}  

\subsection{3-D Shapes}

\subsection{Actions} 

\section{Limitations of Existing Systems}

\subsection{3-D Shapes}

\subsection{Actions}

\subsection{Approaches}

\subsection{3-D Shapes} 

\subsection{Actions}

\subsection{Contributions}

The main contributions of this thesis are:
\begin{itemize}
	\item A detailed performance evaluation on several state-of-the-art 3-D interest point detectors.
	\item A new semi-supervised constellation model for 3-D shape recognition and registration. 
	\item A real-time algorithm that combines appearance and structural information for video-based action categorisation. 
	\item A new technique to estimate 3-D human body poses using action detection and regression random forests. 
	\item A semi-supervised 3-D hand pose estimation algorithm that combines synthetic and realistic training data. 
\end{itemize}

\section{Thesis Outline}

This thesis is composed of two parts. Chapter \ref{chap/lit3d} to Chapter \ref{chap/3dreg} focus on the recognition and pose estimation of 3-D shapes; the second part, from Chapter \ref{chap/lithuman} to \ref{chap/hand}, focuses on human action analysis, including action categorisation and 3-D pose estimation of human body and hand gestures.  

\paragraph{Part I: 3-D Shapes}

\paragraph{Chapter 2.} 
This chapter presents a literature review of 3-D shape processing techniques.  
Existing methods for 3-D feature detection, shape representation, shape recognition and pose estimation are discussed.  

\paragraph{Chapter 3.} 
A performance evaluation of volumetric 3-D interest points is presented in this chapter. 
It first gives a brief introduction to the volumetric interest point detectors used in existing literature.
The interest point detectors are evaluated quantitatively using a new performance metric.
Finally, the characteristics of the candidate detectors are discussed qualitatively. 

\paragraph{Chapter 4.}
This chapter presents a weakly supervised constellation model for simultaneous 3-D shape recognition and pose estimation. 
The proposed model learns the shape, appearance and pose \emph{(SAP)} of an object class from training exemplars, \eg images or point clouds, containing examples of the object in unknown poses.  

\paragraph{Part II: Human Action Analysis}

\paragraph{Chapter 5.} 
This chapter introduces various sub-problems in human action analysis, including human action recognition, body pose estimation and hand pose estimation. 
It also discuss the use of random forests and its variants to model human actions and poses.    
It also reviews the existing approaches for the above sub-problems. 

\paragraph{Chapter 6.} 
This chapter addresses the recognition sub-problem in human action analysis. The spatiotemporal semantic and structural forest is proposed to recognise human actions in real-time.    

\paragraph{Chapter 7.} 
This chapter presents a new approach for 3-D body pose estimation in unconstrained, monoculat videos. .   
A new random forest learning algorithm is introduced, in order to perform action detection and pose estimation simultaneously. 
% By combining regression random forests with action detection, the proposed method infers 3-D poses from unconstrained, monocular videos. 

\paragraph{Chapter 8.} 
This chapter addresses the sub-problem of 3-D hand-pose estimation. 
The regression tree learning algorithm introduced in chapter \ref{chap/body} is further developed to estimate 3-D hand pose from different viewpoints. 
In addition, a data-driven inverse kinematic scheme is described to refine the occluded joints.

