\documentclass[10pt, a4paper]{article}

\usepackage{geometry}
\usepackage{color}

\title{List of corrections \\ {\large Classification and Pose Estimation of 3D Shapes and Human Actions} }
\author{Tsz-Ho Yu}
\date{\today}

\begin{document}

\maketitle

\section{General Changes}
\begin{itemize}
\item All grammatical errors and typos marked by the examinars are corrected, according to the marked draft.
\item Proofreading has been performed manually and with a spellchecker program.
\item All figures are labelled and referenced at least once in the text.
\item The testing dataset for chapter 5 is now available at: http://www.thyu.org/ape/
\item The testing dataset for chapter 6 is now available at: http://www.iis.ee.ic.ac.uk/~dtang/hand.html
\end{itemize}

\section{Abstract}
\begin{itemize}
\item All typos are corrected.
\item Introductions to chapter 5 and 6 are expanded.
\end{itemize}

\section{Declaration}
\begin{itemize}
\item ``Abstract'' are changed to ``paper'' in the declaration.
\end{itemize}

\section{Chapter 1: Introduction}
\begin{itemize}
\item Figure 1.1 has been revised.
\item The thesis outline (section 1.2.3) has been expanded.
\end{itemize}

\section{Chapter 2: Evaluation of 3D Feature Detectors}

\begin{itemize}
\item Cosmetic improvements to Table 2.1: fixed missing table border in the first row
\item Cosmetic improvements to Table 2.2: revised the table headers
\item Figure 2.4 is made bigger.
\item The captions of figure 2.5, 2.6 have been revised.
\item Table 2.3 has been rotated to utilise full page space. 
\item The caption of figure 1.10 is edited slightly.
\item In section 2.1, added a reference \cite{Hadfield2013} as an example of treating depth images as spatiotemporal volumes.
\item In section 2.2.1, added a reference \cite{Hadfield2013} in both corner detector (for 3.5D/4D Harris) and blob detector (for 3.5D/4D Hessian)
\item Caption of figure 2.8 has been revised.
\end{itemize}

\section{Chapter 3: 3D Constellation Model From Unknown Poses}
\begin{itemize}
\item The explanation of a constellation model (``A constellation model is...'') has been expanded and moved from section 3.3.1 to the end of section 3.1. 
\item A new figure 3.1 is added as an overview of the proposed approach. 
\item Table 3.1 is added to list the random variables used in the graphical model.
\item The subfigures in figure 3.2 become bigger, more examples has been added to figure 3.2(a); the caption of figure 3.2 is revised.
\item The caption of figure 3.3 is edited. The layout of figure 3.3 is revised.
\item Table 3.2 is added to provide more information about the random poses used in the experiments.
\item Table 3.3 is revised:
	\begin{itemize}
	\item Definition of K and Q are added to the caption.
	\item The layout of table 3.3 is rotated.
	\end{itemize}
\item Figure 3.4 (the Bike dataset) is corrected.
\item Figure 3.5 is made bigger.
\item Implementation details (feature detector and feature descriptor) of the Toshiba point cloud dataset has been added.
\item For the method evaluated in figure 3.6, a short introduction to Hough transform and its variants is added, e.g. \cite{Woodford2013} and \cite{Barinova2010}, in the ``related work'' section.
\item The explanation of figure 3.6 is revised.
\item Limitations of the proposed method is added in the summary section.
\end{itemize}

\section{Chapter 4: Human Action Classification}

\begin{itemize}
\item ``look-ahead time'' is renamed to ``recognition latency'' for clarity.
\item Figure 4.2 (Overview of the proposed approach) has been redrawn. The caption of figure 4.2 is revised.
\item Section 4.3 is edited to explain figure 4.2.
\item Figure 4.3 has been redrawn. The caption of figure 4.3 is revised.
\item Section 4.4 is edited to explain figure 4.3.
\item In table 4.1,  ``relative running time'' is used instead of ``relative speed'', speedup factor is added to the table for reference.
\item Figure 4.5 has been redrawn to explain the 3D histograms of the HSRM algorithm.
\item The text explaining figure 4.5 is edited.
\item Introduction to the origin SRM algorithm \cite{Ryoo2009} is added in section 4.6.
\item Figure 4.6 is added to illustrate pyramid match kernel for the HSRM histogram.
\item Description of figure 4.6 is added in the text.  
\item Add limitations of the proposed action classification system.  
\item Subsection 4.7.1 ``Bag of semantic textons'' is promoted to a section, the original section titled ``Combined classification with bag-of-semantic-textons'' is removed.
\item Subsection 4.7.2 is promoted to a section and renamed to ``Late fusion classificaiton''.
\item Parameter $\alpha$ in the late fusion scheme is explained in the text.
\item In table 4.4, millisecond-per-frame is usd instead of frame-per-second to indicate the run-time performances of different processes. Total average FPS is also provided in the table.
\end{itemize}

\section{Chapter 5: 3D Human Body Estimation}

\begin{itemize}
\item Figure 5.1 has been re-designed.
\item The old figure 5.2 is removed as it is not related to the knowledge transfer ability of the proposed method.
\item Figure 5.2 (the overview) has been redrawn.
\item The description of figure 5.2 in the text is edited.
\item Figure 5.3 is revised, the variable names are corrected.
\item The description of figure 5.3 (graphical model, the old figure 5.2) in the text is edited.
\item In section 5.3, the reason of not using the HSRM algorithm (chapter 4) is explained.
\item The legend of figure 5.5 and 5.7 are revised.
\item Figure 5.6 is added to show the joint names (joint labels) used in the experiment.
\end{itemize}

\section{Chapter 6: 3D Hand Pose Estimation} 

\begin{itemize}
\item In section 6.2.1, ``generative'' and ``discriminative'' approaches are renamed to ``model-based'' and ``appearance-based'' respectively, according to the literature review by Erol et al. \cite{Erol2007}.
\item The description of the apperance-based methods is revised and updated.  
\item Figure 6.2 has been redrawn to explain the proposed framework clearly.
\item Description of figure 6.2 in the text (section 6.3) are revised.
\item In section 6.4.2 ``viewpoint classification term'', the reservoir sampling method is explained. (It is actually ``random sampling without replacement''), \cite{Vitter1985} is added to the text as a reference.
\item In section 6.4.2, the adaptive switching paramters is explained. 
\item Captions are added to figures 6.4, 6.5 and 6.6.
\item In section 6.7, limitations of the proposed hand pose estimation model is discussed.
\end{itemize}

\section{Chapter 7: Conclusion}
\begin{itemize}
\item Future work is revised and expanded substantially. 
\end{itemize}

\bibliographystyle{apalike}
\bibliography{references} 

\end{document}
