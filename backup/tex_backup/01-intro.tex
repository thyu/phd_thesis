\chapter{Introduction}
\label{chap/intro} 

Three dimensional object recognition is a fundamental problem in computer vision. It is encompasses the detection, identification, classificaiton and pose estimation of objects from 3-D and time-varying 2-D visual data in general, including point clouds, depth maps, voxels and videos. 

The topics of 3-D object recognition were studied extensively since the early days of computer vision research. Early approaches of 3-D shape recognition are detailed in the literature review in \cite{Besl1985}. Early techniques of video-based human motion classification and pose estimation are reviewed in \cite{Cedras1995} and \cite{Aggarwal1999}. 
% Pioneer articulated very different framework for representing and recognising 3-D objects. 
Pioneers of computer vision have proposed various frameworks for representing and recognising objects from 3-D data.
% The number of literature reflected how the early researher treat the topic seriously. 
% Marr 
For instance, \cite{Marr1978} presented the primal sketch, in which an 3-D object is represented as a hierarchical set of cylinders in an object-centric frame. 
% Generalised cone by Nevatia 
In addition, \cite{Nevatia1977} proposed the generalised cones framework to describe and recognise 3-D objects. \cite{Bolles1983} studied the pose estimation of 3-D objects in depth maps.
% Moving light display 
Concerning object recognition in video data, \cite{Johansson1973} pioneered video-based action recognition using a moving light display. 
% Nevertheless, practical applications of such early models are limited, due to the restrictions in computational power to process large amount of 3-D data and the high cost of collecting sufficient data for training and testing.
The practical applications of these early models are limited: Firstly, the computational power available were inadequate to process large amount of 3-D data for sophisicated algorithms. Secondly, it was costly to collect sufficient data for training and testing, without efficient 3-D data capturing devices and technologies. Finally, they did not consider occlusions, viewpoints or pose variations which are archetypal issues in practical object recognition applications. Image-based object recognition, being more efficient and attainable, has gradually gained more attention, because it is more applicable to practical object recognition systems than its 3-D counterpart by then.   

However, research interest in 3-D object recognition has been rekindled recently. 
With the advent of affordable data acquistiion technologies, such as depth sensors \cite{Shotton2011}, multi-view stereo systems \cite{Vogiatzis2011} and video cameras in mobile devices, 3-D data have become more easily accessible. Meanwhile, the availability of large-scale 3-D datasets have made automatic 3-D object recognition possible using state-of-the-art machine learning techniques.  

As a result, this thesis addresses the sub-problems of \emph{classification} and \emph{pose-estimation} for \emph{3-D shapes} and \emph{human actions in videos}. 
% Videos and 3-D shape data are the most accessible representation of 3-D object data.
Videos and 3-D shapes are the most accessible representation of 3-D spatial and spatiotemporal objects, which can be captured from video cameras and affordable modern 3-D shape acquisition devices, e.g. multi-view stereo \cite{Vogiatzis2011}. 
% Photometric textures on 3-D shape surfaces are not considered because they are more costly and difficult to be captured.  
In addition, large-scale benchmark datasets, e.g. Princeton shape benchmark \cite{Shilane2004},  KTH action dataset\cite{Schuldt2004}, TOSCA dataset \cite{Bronstein2011} and Youtube video dataset \cite{Liu2009}, have been made publicly available, facilitiating standardised comparison among different approaches. 
On the other side, classification and pose estimation are the essential components of various applications, such as human-computer interface, computer-aided design, automatic fault detection, video surveillance, computer graphics and entertainment. 
% Why classification? Maybe add some more\ldots 
% Why pose estimation? Matbe add some more\ldots

This thesis is organised into two parts.  
The first part is concerned with the classification and pose estimation of 3-D shape data. It starts with a performance evaluation of interest point detectors for 3-D shape data. It then proposes a new semi-supervised deformable part model which performs 3-D shape classification and pose estimation simultaneously. In the second part of this thesis, three new algorithms for human action classification, 3-D body pose estimation and 3-D hannd pose estimation from videos and time-varying depth images are proposed. Experimental results have demonstrated that state-of-the-art performances can be achieved using various random forest algorithms.    

\section{Challenges}

% problems of 3-D shape recognition
Although 3-D classification and pose estimation have been studied for decades, there is still much room for improvement in fully-automatic shape or action recognition systems. 
This thesis is concerned with some of the long-standing challenges of 3-D shapes and video recognition tasks.  

\subsection{3-D Shape Classification and Registration} 

Different 3-D shape representations lead to diversified features for classification and registration. Designing an generic 3-D shape recognition algorithm is difficult. Many existing methods are restricted to certain data representations, especially triangular mesh which is a standard format for synthetic shape data, e.g. Harris3D @@ Sipiran @@ and MeshHOG @@ Zaharescu @@.     
While 2-D image-based interest points are studied extensively, e.g. @@ Schimid survey @@, there are few performance evaluations for 3-D interest points. 
This thesis is mainly concerned with the interest poinsts in volumetric data, which is a generic representation of 3-D and time-varying data. Voxels can be converted efficiently from other data formats, such as point clouds, meshes and videos. 

In addition, categorical pose estimation of 3-D shapes has been a long standing issue in computer vision.
Standard bag-of-words models for 3-D object categorisation do not infer object pose, as they disregard the structural information of objects. Traditional 3-D registration algorithms, such as iterative closest point (ICP) @@ @@, do not model the variations among instances in the same object class. 
Generative part-based models, e.g. @@ Fergus @@, provide an supervised framework to infer both scale and translation of an object class. However, object rotations are not estimated.   
Voting-based methods @@ @@ performs 3-D shape classification and registration simultaneously using generalised Hough transform. Still, they require training data to be registered manually in advance.  

\subsection{Human Action Classification}

% Computational bottleneck

Although bag-of-words is an established approach for easlier video classification tasks @@ @@, its performance is limited because the codeword histograms utilise no structural information. Run-time performance is also a major issue in human action classification. Although real-time action classification is essential for many applications, such as human-computer interfaces and video surveillance, existing methods perform classification only after the whole testing sequence has been processed @@ some examples @@. 

\subsection{3-D human pose estimation} 

% uncontrolled environments
% 1. Background
% 2. Special Hardware, multiple cameras  

Recent 3-D human pose estimation algorithms achieve promising performances in controlled environments @@ @@. On the other side, 3-D human pose estimation techniques in unconstrained environments are largely unexplored.       
Traditional pose estimation algorithms often rely on low-level appearance features, such as silhouette and motion templates @@ @@. These features are vulnerable to non-static backgrounds and moving viewpoints in unconstrained videos. 
Pose ambiguities and self-occlusions are common issues in both body pose and hand pose estimation. While multi-view techniques resolve such problems effectively in controlled scenes, they are not applicable to unconstrained, monocular testing videos. Additional knokwledge about the pose is required to resolve the aforementioned ambiguities in monocular videos. 

\section{Thesis overview}

\subsection{Contributions}

The main contributions of this thesis are:
\begin{itemize}
	\item A detailed performance evaluation on several state-of-the-art 3-D interest point detectors.
	\item A new semi-supervised constellation model for 3-D shape recognition and registration. 
	\item A real-time algorithm that combines appearance and structural information for video-based action categorisation. 
	\item A new technique to estimate 3-D human body poses using action detection and regression random forests. 
	\item A semi-supervised 3-D hand pose estimation algorithm that combines synthetic and realistic training data. 
\end{itemize}

\subsection{Limiting the scope}

% Some thing which is not considered?
Several applications are not considered within this thesis. 
The first is recognition of textured 3-D shapes @@ @@. Textured meshes is a standard representation of 3-D shapes in computer graphics and computer-aided design (CAD) tasks. However, realistic 3-D shape data are usually captured in textureless point clouds, depthmaps or voxels. In addition, ;instances of the same object class can have different textures, texture-based features are therefore not suitable for shape classification.      

The second application is model-based 3-D object recognitions, e.g. @@ @@. These approaches are trained on object instances in a specific scene. The performances of object recognition and pose estimation rely largely on the process of feature matching between the model and query object instance. 

Thirdly, marker-based motion capture sysems are not included in the scope of this thesis, @@ @@. They have been widely employed in computer-generated imagery for films and games. Markers are localised accurately from a calibrated camera system, using simple computer vision techqniues. Poses are estimated straightforwardly by mapping the markers to a predefined articulated skeleton.

Finally, body shape estimation from images and videos are not considered in the thesis, @@ @@. It reconstructs 3-D human shapes directly from testing data, instead of parameterised poses of a articulated model. The task of 3-D body pose estimation is concerned with understanding the subject in the scene, rather than reconstructing it accurately. 

\subsection{Thesis Outline}

This thesis is composed of two parts. Chapter \ref{chap/lit3d} to Chapter \ref{chap/3dreg} focus on the recognition and pose estimation of 3-D shapes; the second part, from Chapter \ref{chap/lithuman} to \ref{chap/hand}, focuses on human action analysis, including action categorisation and 3-D pose estimation of human body and hand gestures.  

\paragraph{Part I: 3-D Shapes}

\paragraph{Chapter 2.} 
This chapter presents a literature review of 3-D shape processing techniques.  
Existing methods for 3-D feature detection, shape representation, shape recognition and pose estimation are discussed.  

\paragraph{Chapter 3.} 
A performance evaluation of volumetric 3-D interest points is presented in this chapter. 
It first gives a brief introduction to the volumetric interest point detectors used in existing literature.
The interest point detectors are evaluated quantitatively using a new performance metric.
Finally, the characteristics of the candidate detectors are discussed qualitatively. 

\paragraph{Chapter 4.}
This chapter presents a weakly supervised constellation model for simultaneous 3-D shape recognition and pose estimation. 
The proposed model learns the shape, appearance and pose \emph{(SAP)} of an object class from training exemplars, \eg images or point clouds, containing examples of the object in unknown poses.  

\paragraph{Part II: Human Action Analysis}

\paragraph{Chapter 5.} 
This chapter introduces various sub-problems in human action analysis, including human action recognition, body pose estimation and hand pose estimation. 
It also discuss the use of random forests and its variants to model human actions and poses.    
It also reviews the existing approaches for the above sub-problems. 

\paragraph{Chapter 6.} 
This chapter addresses the recognition sub-problem in human action analysis. The spatiotemporal semantic and structural forest is proposed to recognise human actions in real-time.    

\paragraph{Chapter 7.} 
This chapter presents a new approach for 3-D body pose estimation in unconstrained, monoculat videos. .   
A new random forest learning algorithm is introduced, in order to perform action detection and pose estimation simultaneously. 
% By combining regression random forests with action detection, the proposed method infers 3-D poses from unconstrained, monocular videos. 

\paragraph{Chapter 8.} 
This chapter addresses the sub-problem of 3-D hand-pose estimation. 
The regression tree learning algorithm introduced in chapter \ref{chap/body} is further developed to estimate 3-D hand pose from different viewpoints. 
In addition, a data-driven inverse kinematic scheme is described to refine the occluded joints.

