\documentclass[10pt, a4paper]{article}
\usepackage{setspace}
\onehalfspace

\begin{document}
 ----------------------------------------------------------------

Three dimensional object recognition is a fundamental problem in computer vision. It is encompasses the detection, identification, classificaiton and pose estimation of objects from 3-D and time-varying 2-D visual data, including point clouds, depth maps, voxels and videos. 

The topics of 3-D object recognition were studied extensively since the early days of computer vision research. Early approaches of 3-D object recognition are detailed in the literature review in \cite{Besl1985}. 
% Pioneer articulated very different framework for representing and recognising 3-D objects. 
Pioneers of computer vision have articulated various frameworks for representing and recognising objects from 3-D data.
% The number of literature reflected how the early researher treat the topic seriously. 
% Marr 
For instance, \cite{Marr1978} presented the primal sketch, in which an 3-D object is represented as a hierarchical set of cylinders in an object-centric frame. 
% Generalised cone by Nevatia 
In addition, \cite{Nevatia1977} proposed the generalised cones framework to describe and recognise 3-D objects. \cite{Bolles1983} studied the pose estimation of 3-D objects in depth maps.
% Moving light display 
Concerning object recognition in video data, \cite{Johansson1973} pioneered video-based action recognition using a moving light display. Nevertheless, practical applications of such early models are limited, due to the restrictions in computational power to process large amount of 3-D data and the high cost of collecting sufficient data for training and testing. 

% Image-based recognition systems have taken place. 

% The literature review of early 3-D object recognition approaches are detailed in \cite{Besl1985}. 

Research interest in 3-D object recognition has been rekindled recently.    
With the advent of affordable data acquistiion technologies, such as depth sensors \cite{Shotton2011}, multi-view stereo systems \cite{Vogiatzis2011} and video cameras in mobile devices, 3-D data have become more easily accessible. Meanwhile, the availability of large-scale 3-D datasets have made automatic 3-D object recognition possible using state-of-the-art machine learning techniques.  

This thesis addresses the sub-problems of \emph{classification} and \emph{pose-estimation} for \emph{3-D shapes} and \emph{human actions in videos}. 
% Videos and 3-D shape data are the most accessible representation of 3-D object data.
Videos and 3-D shape are more accessible than other representations of 3-D object data, which can be captured from video cameras and affordable structured light sensors.    
Large-scale benchmark datasets, e.g. Princeton shape benchmark \cite{Shilane2004},  KTH action dataset\cite{Schuldt2004}, TOSCA dataset \cite{Bronstein2011} and Youtube video dataset \cite{Liu2009}, have been made publicly available, facilitiating standardised quantitative comparison among different approaches. 
On the other side, classification and pose estimation are important in automatic 3-D object recognition system. They are required by numerous applications, such as human-computer interface, computer-aided design, automatic fault detection, video surveillance, computer graphics and entertainment. 
% Why classification? Maybe add some more\ldots 
% Why pose estimation? Matbe add some more\ldots

This thesis comprises two parts. 
The first part is concerned with the classification and pose estimation of 3-D shape data.   
A performance evaluation of volumetric interest point detector for 3-D shapes is presented.  

In the second part, human action categorisation and 3-D human pose estimation are discussed. Three different new algorithm are proposed for action classification, 3-D body pose estimation and 3-D hand pose estimation respectively. 

\section{Challenges}

% problems of 3-D shape recognition

Although 3-D classification and pose estimation have been studied for decades, there is still much room for improvement in fully-automatic shape or action recognition systems. 
This thesis focuses on tackling the long-standing challenges encountered by 3-D shapes and video recognition tasks.  

\paragraph{3-D Shapes} 

% Model-wise: BOW model does not consider structural information of objects. Many approaches work on a specific type of 3-D shape representation, such as mesh.      

Scalability is one of the major issues in 3-D shape recognition. Standard method for processing images will become too computationally demanding for processing much larger 3-D shape data.  

% 3D interest point detection
% - Many of them use data specific properties 
% - There lacks a unified framework for evaluating interest points. 
% - There exist many different kinds of 3-D features, mostly a repurposed version of 2-D feature detector, 

Different representations of 3-D shape data lead to a diversified range of 3-D feature detectors and descriptors. Some of them are only applicable to certain data representations, such as textured mesh \cite{} and range images \cite{}. 
We are more concered with the volumetric interest point detection because it is more generic and applicable to both shape and video data. 
Interest point detectors for volumetric data, e.g. voxels and videos, are often repurposed from their image-based (2-D) counterparts. While the performances of image-based interest points have been evaluated comprehensively, volumetric data  

% 3D shape recognition
% - Global based --- not good for incomplete shape 
% - No structural information
% - Not scalable -- need to register shape manually before training

Many existing solutions for 3-D shape recognition are not scalable. 
They require registered training instances which involves manual work. 

Most existing approaches leverages properties of a particular data representation, such as triangular mesh \cite{ } and range images \cite{ }.   

\paragraph{Human Action Classification}

% Computational bottle-neck of feature extraction
% Efficient classification method 
% Background subtraction
% Agile action categorisation: perform robust action recognition using just a few frames  
Human action .   

\paragraph{3-D Human Pose Estimation} 

The ``bag-of-words'' model is a standard framework for many video-based action classification methods. 
The computational bottleneck of action recognition 
Existing 3-D human pose estimation algorithms usually use low-level appearance features like silhouette or motion templates, which are not applicable to uncontrained environment.   

----------------------------------------------------------------

Processing of training data 

Unconstrained classification and pose estimation 


This thesis addresses the problem of recognition and pose estimation of 3-D data including 3-D shapes and videos.  Object recognition and pose estimation are two fundamental challenges in computer vision.  3-D shapes and human actions analysis are the two central topics in current computer vision research. 

Object recognition and pose recognitions are two fundamental problems in computer vision research. 

They have been studied since the early days of computer vision, when pioneers of the field conceptualised various representations.

They have been studied since the early days of computer vision, while pioneers in the field conceptualised different methods to describe a 3-D object, e.g. the cylindrical part model by Marr and generalised cone model by Binford for 3D shapes, and adfadfdfsdaf for videos.    

In the early days of computer vision research, practical applications of 3-D object recognition were largely limited because of the constraints in computational power and cost in data acquisition.  emerges again as a popular area of research.    

Agin and Binfor Generalised Cone
Marr 3D-model sketch - Hierarhical Part Model --- mental rotation is required because only one view is stored in memory 



\bibliographystyle{apalike} 
\bibliography{references}
\end{document} 
