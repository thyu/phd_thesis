\chapter{Human Action Analysis: Literature Review}
\label{chap/lithuman}

\section{Limitations of Existing Approaches} 

\subsection{Action Recognition}



The problem of human activity analysis has been a long standing challenge of computer vision for decades. 
This thesis addresses the sub-problems of human action analysis.  
which involves retrieval of high-level semantics from visual data. 
Human Action Analysis

- Movement
- Action category   
- Pose
- Gesture
- Interaction with objects
- Interactions with other people

\paragraph{Problems of Existing Approaches}

\paragraph{What we do?} 

\paragraph{Why random forest?} 

Unlike object recognition tasks on 2-D images, spatiotemporal analysis of human action is more computationally intensive, due to the much larger training and testing spatiotemporal data (video, depth image sequences) involved. 
As a result, random forest emerges as a feasible candidate for learning and testing for video-based human action analysis.
Random forest is recognised as a efficient techniques for computer vision task, it has already been applied in many different kinds of applications, such as image recognition, keypoint tracking and semantic segmentation. 

Random forest is considered as a feasible framework for human action analysis becuase:
Firstly, randomly forest is very efficient. 
Secondly, random forest handles multi-class data well. 
Thirdly, random forest itself is a framework that serves multiple purposes. Different random forest learning algorithms have been proposed to suit various kinds of computer vision applications, \eg  

Pose estimation: background 

\section{Our Contributions}

Action recognition: we proposed a new approach to quickly recognize actions in real time. We use both appearance and structural information to infer the action category. 

3-D Body Pose Estimation: instead of using low-level appearance features, we combines the DPM technique which is widely used in 2-D Pose Estimation. We proposed 

3-D Hand Pose Estimation: addressing the difficulties of hand-labeling complicated joint labels, we proposed a semi-supervised algorithm that learns the .

\section{Thesis Outline} 
